\documentclass{/workdir/classes/summary}

\graphicspath{{figures/}}

% タイトルの項目
\title{スペクトル画像によるコーン指数推定}
\etitle{Cone Index Estimation with Spectral Images}
\studentid{03-210401}
\author{◯◯ ◯◯}
\supervisor{◯◯ ◯◯ 教授}
\abst{
    災害現場における無人化施工では軟弱地盤による建設機械のスタック等を防ぐために地盤強度を把握することが重要である.
    本研究では,光源やカメラの位置関係を考慮した反射モデルを用いて分光反射率を推定し,この分光反射率から地盤強度を表す指標であるコーン指数を推定する手法を提案する.
    検証実験の結果から,地表面に対する光源やカメラの位置関係を考慮したコーン指数を推定する手法の有効性が確認できた.
}

\begin{document}
\maketitle

\section{序論}
日本は世界に比べて自然災害が高い頻度で発生し,このような自然災害は土砂災害を誘発する場合が多い.
土砂災害の復旧作業には建設機械が欠かせないが,災害現場では作業員が二次災害に巻き込まれる危険性がある.
そのため,建設機械を遠隔操作する無人化施工の需要が高まっている.
しかし,無人化施工では操縦者の体性感覚に頼れないため,軟弱な地盤でのスタック等の回避を操作中に行うことが困難になる.
そこで,地盤強度を調査し,建設機械が走行可能な範囲をあらかじめ把握することが考えられる.
この調査は建設機械を投入する前に行われる必要があるので,迅速さが求められる.

筑紫らはスペクトル画像を用いて地盤の強度を表すコーン指数を推定する手法を提案している\cite{chikushi2020}.
この手法では画像を用いているため,一度に広範囲の地盤を測定対象とすることができる.
分光反射率が既知である校正用シートと地盤のスペクトルを比較して地盤の分光反射率を求め,推定に用いているが,地表面に対する光源やカメラの位置関係は考慮されていない.
一般に,物体に対する光源や観測の幾何的な位置関係によって,観測される反射光強度が変化する.
そのため,凹凸や傾斜の変化などで位置関係に変化が生じると,反射光強度の変化によって分光反射率が正しく推定されず,コーン指数推定の信頼性も低下する.
そこで本研究の目的を,凹凸や傾斜の変化に頑健なコーン指数推定手法の提案とする.

\section{提案手法}
\subsection{対象とするシステムの概要}

\subsection{分光反射率の推定}

\subsection{コーン指数の推定}

\section{検証実験}

\section{実験結果・考察}
\subsection{分光反射率推定}

\subsection{コーン指数推定}

\section{結論}

\bibliographystyle{pesummary}
\bibliography{sections/reference}
\end{document}
